\newpage
\chapter{INTRODUCTION}
Java is a General Purpose Programming Language (GPL) and also powerful programming language.
Java developed at \textbf{Sun Microsystems} which was purchased by \textbf{Oracle} in 2010.
Java is GPL because it is used to solve a wide variety of problems and build software.

%%%%%%%%%%%%%%%%%%%%%%%%%%% Specification %%%%%%%%%%%%%%%%%%%%%%%%%%%%%%%
\section{JAVA LANGUAGE SPECIFICATION}

%%%%%%%%%%%%%%%%%%%%%%%%%%%%%%%%%%%%%
\subsection{THE SYNTAX \& SEMANTICS}
To write English we should follow some rules (Grammar). Also, to write java we should follow some rules that is called \textbf{syntax \& semantics}\\
\textbf{Example:}
\begin{itemize}
	\item[{\LARGE $\diamond$}] He \textcolor{red}{are} playing $\Rightarrow$ Syntax error.(Grammar)
	\item[{\LARGE $\diamond$}] He is hello and bye $\Rightarrow$ Semantic error.(Meaning)
\end{itemize}


%%%%%%%%%%%%%%%%%%%%%%%%%%%%%%%%%%%%%%%
\subsection{API}
\textbf{Application Programming Interface (API)} also known as \textbf{library}. It contains predefined Java code that we can use to develop Java programs. It makes faster and easier development process. Because we do not need to write everything from scratch.

%%%%%%%%%%%%%%%%%%%%%%%%%%%%%%%%%%%%%%%
\subsection{EDITIONS OF JAVA}
Java comes in three editions
\begin{itemize}
	\item[{\LARGE $\diamond$}] \textbf{Standard Edition(SE):} Develop applications that run on desktop.
	\item[{\LARGE $\diamond$}] \textbf{Enterprise Edition(EE):} Develop server-side applications.
	\item[{\LARGE $\diamond$}] \textbf{Micro Edition(ME):} Develop applications for mobile devices.
\end{itemize}

\begin{quotation}
	\textbf{NOTE:} Java SE is the foundation of all other editions.
\end{quotation}


%%%%%%%%%%%%%%%%%%%%%%%%%
\subsection{JDK}
Java Development Kit (JDK)
\begin{itemize}
	\item[{\LARGE $\diamond$}] Set of programs that enable us to develop our programs.
	
	\item[{\LARGE $\diamond$}] Contains \textbf{JRE(Java Runtime Environment)} that is used to run out programs.
	
	\item[{\LARGE $\diamond$}] \textbf{JDK \& JRE} contain \textbf{JVM (Java Virtual Machine)}.
	
	\item[{\LARGE $\diamond$}] \textbf{JVM} executes our java programs on different machines that makes Java independent.
\end{itemize}

%%%%%%%%%%%%%%%%%%%%%%%%%%%%%%%%
\subsection{IDE}
\textbf{Integrated Development Environment (IDE)} is a program that allows us to-
\begin{itemize}
	\item[{\LARGE $\diamond$}] \textbf{Write}: Write source code
	\item[{\LARGE $\diamond$}] \textbf{Compile}: Translate source code to machine code
	\item[{\LARGE $\diamond$}] \textbf{Debug}: Tools to find errors
	\item[{\LARGE $\diamond$}] \textbf{Build}: Files that can be executed by JVM
	\item[{\LARGE $\diamond$}] \textbf{Run}: Execute program
\end{itemize}
IDE makes development faster and easier. NetBeans, Eclipse, IntelliJ IDE are the popular Java IDEs.
\begin{quotation}
	\textbf{NOTE:} The Java source code first compiled into a binary byte code using Java compiler, then this byte code runs on the JVM, Which is a software based interpreter. So Java is considered as both interpreted and compiled.
\end{quotation} 


%%%%%%%%%%%%%%%%%%%%%%% Anatomy of Java Program %%%%%%%%%%%%%%%
\section{ANATOMY OF JAVA PROGRAM}

%%%%%%%%%%%%%
\subsection{CLASS}
A blueprint to create \textsl{OBJECTS}.

%%%%%%
\subsubsection{CLASS STRUCTURE}
\begin{lstlisting}[language=java]
	class class_name {
		code block
	}
	// "class" is keyword.
\end{lstlisting}

%%%%%%%%%%%%%%%
\subsection{OBJECTS}
An instance of a \textsl{CLASS}.

%%%%%%%%%%%%%%%
\subsection{METHOD}
Group of instruction to do a specific task.

\subsubsection{METHOD STRUCTURE}
Each method consists of 4 main parts.
\begin{itemize}
	\item[{\LARGE $\diamond$}] Return Type
	\item[{\LARGE $\diamond$}] Method Name
	\item[{\LARGE $\diamond$}]Parameter
	\item[{\LARGE $\diamond$}]Code Block
\end{itemize}

\begin{lstlisting}[language=java]
	return_type method_name(parameter) {
		code block
	}
\end{lstlisting}

\begin{quotation}
	\textbf{NOTE :} Every method is written inside a \textsl{CLASS}.
\end{quotation}

\subsubsection{CALLING A METHOD}
It is basically using the method
\begin{lstlisting}[language=java]
	method_name(parameter);
\end{lstlisting}

\begin{quotation}
	\textbf{NOTE :} The \textsl{main()} method is automatically called when we run the JAVA program.
	\begin{itemize}
		\item It is the first method that is called.
		\item It is the starting point of execution of a program.
	\end{itemize}
\end{quotation}

%%%%%%%%%%%%%%%%%%%%%%%%%%%%
\subsection{ACCESS MODIFIERS}
The access modifiers in JAVA specifies the accessibility of a field, method, constructor, or class.\\
There are four types of JAVA access modifiers :
\begin{itemize}
	\item[{\LARGE $\diamond$}] \textbf{Private :} Access level only within the class, can not be accessed from outside the class.
	
	\item[{\LARGE $\diamond$}] \textbf{Default :} Access level only within the package. If do not specify any access leve then it will be the default.
	
	\item[{\LARGE $\diamond$}] \textbf{Protected :} Access level within the package and outside the package \textsl{through child class}. If do not make the child class, it can not be accessed from outside the package.
	
	\item[{\LARGE $\diamond$}] \textbf{Public :} Access level everywhere, within or outside the class and package.
\end{itemize}


%%%%%%%%%%%%%%%%%%%%%%%%%%%
\subsection{NAMING CONVENTIONS}
How to write name in programming.
\begin{itemize}
	\item[{\LARGE $\diamond$}] \textbf{Pascal Case Convention :}
	\begin{itemize}
		\item ThisIsAName
		\item Naming \textsl{Class}
	\end{itemize}
		
		
	\item[{\LARGE $\diamond$}] \textbf{Camel Case Convention :}
	\begin{itemize}
		\item thisIsAName
		\item Naming \textsl{Methods \& Variables}
	\end{itemize}
	
	\item[{\LARGE $\diamond$}] \textbf{Snake Case Convention :}
	\begin{itemize}
		\item this\_is\_name
	\end{itemize}
	
\end{itemize}


%%%%%%%%%%%%%%%%%%%%%%%%%%%%
\subsection{JAVA PROGRAM STRUCTURE}
\begin{lstlisting}[language=java]
	public class Main {
		public static void main ( String[] args) {
			code_block
		}
	}
\end{lstlisting}

%%%%%%%%%%%%%%%%%
\subsection{PACKAGE}
A container for Classes.

\begin{quotation}
	\textbf{NOTE :}
	\begin{itemize}
		\item Package contains Classes
		\item Class contains Methods \&
		\item Method contains code blocks
	\end{itemize}
\end{quotation}