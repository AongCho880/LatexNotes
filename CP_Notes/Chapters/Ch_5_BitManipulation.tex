\newpage
\chapter{Bit-Manipulation Basics}

%%%%%%%%%%%%%%%%%
\section{\textbf{Binary AND}}
\textbf{Check whether a number is odd or even}
\begin{lstlisting}[language=c++]

bool isOdd(int n) {
    return (n&1) ? true : false;
}

\end{lstlisting}

%%%%%%%%%%%%% Bit Shifting %%%%%%%%%%%%%%%
\section{\textbf{Bit Shifting}}

%%%%%%%%%
\subsection{Binary Left Shift}
\tab \texttt{ $a << b = a*2^b$}

\begin{lstlisting}[language=c++]

#include <iostream>
#include <cmath>
using namespace std;

int main() {
    int a=5,b=2;
    cout << (a<<b) << endl;
    cout << a*pow(2,b) << endl;  
    return 0;
}

\end{lstlisting}


%%%%%%%%%%%%%%%%%%%
\subsection{Binary Right Shift}
\tab \texttt{$ a >> b = a/(2^b)$}

\begin{lstlisting}[language=c++]

int main() {
    int a=5,b=2;
    cout << (a>>b) << endl;
    cout << floor(a/pow(2,b)) << endl;  
    return 0;
}

\end{lstlisting}

\subsection{Get $i^{th}$ Bit of an Integer}
\begin{lstlisting}[language=c++]
int getIthBit(int n, int i) {
    return (n & (1<<i)) ? 1 : 0;
}
\end{lstlisting}

\subsection{Clear $i^{th}$ Bit of an Integer}
\begin{lstlisting}[language=c++]
void clearIthBit(int &n, int i) {
    int mask = ~(1<<i);
    n = n & mask;
}
\end{lstlisting}

\subsection{Clear last i Bits of an Integer}
\begin{lstlisting}[language=c++]
void clearLastBits(int &n, int i) {
    int mask = (-1 << i);
    n = n & mask;
}
\end{lstlisting}

\subsection{Set $i^{th}$ Bit of an Integer}
\begin{lstlisting}[language=c++]
void setIthBit(int &n, int i) {
    int mask = (1<<i);
    n = (n | mask);
}
\end{lstlisting}

\subsection{Update $i^{th}$ Bit with value v}
\begin{lstlisting}[language=c++]
void updateIthBit(int &n, int i, int v) {
    int mask = ~(1<<i);
    // Clear ith value first
    n = n & mask;
    // Set ith bit with value v
    mask = (v << i);
    n = n | mask;
}
\end{lstlisting}

\subsection{Clear Bits in a Range}
\begin{align*}
	input:\tab & 11111111\\
	output:\tab & 11100011\\
	\textsl{here, we can divide the } & \textsl{expected out into 2 components}\\
	a:\tab & 111\overbrace{00000}^\text{j+1}\\
	b:\tab & 000000\overbrace{11}^\text{i+1}\tab\Rightarrow\text{$2^i-1$ or $(1<<i)-1$} \\
	\hline
	\textsl{performing bitwise or$(\mid)$ } a\mid b:\tab & 11100011\\
\end{align*}
\textbf{Code}
\begin{lstlisting}[language=c++]
void clearBitsInRange(int &n, int i, int j) {
    // make a = 11100000
    int a = (~0) << (j+1);
    // make b = 00000011
    int b = (1<<i)-1;
    // make mask = 11100011
    int mask = a | b;
    // now clear the bits in between range[i,j]
    n = n & mask;
}

\end{lstlisting}
 %%%%%%%%%%%%%%%%%%%% Power of 2 %%%%%%%%%%%%%%%
\subsection{Check whether a number is power of 2}
\setlength{\columnsep}{3cm}
\begin{multicols}{2}
\begin{center}
\textbf{For power of 2}
\begin{align*}
	2^3 =&\hspace{4pt} 8\\
	(8)_2 =&\hspace{4pt} 00001000\\
	(7)_2 =&\hspace{4pt} 00000111\\
	\hline
	\text{bitwise AND}\hspace{4pt} 8 \& 7 =&\hspace{4pt} 00000000
\end{align*}
\end{center}

\columnbreak

\begin{center}
\textbf{For non power of 2}
\begin{align*}
	(10)_2 =&\hspace{4pt} 00001010\\
	(9)_2 =&\hspace{4pt} 00001001\\
	\hline
	\text{bitwise AND $10 \& 9$} =&\hspace{4pt} 00001000
\end{align*}
\end{center}

\end{multicols}

\textbf{Code Implementation:}
\begin{lstlisting}[language=c++]

bool isPowerOfTwo(int n) {
    return (n & (n-1)) ? false : true;
}

int main() {
    int n; cin >> n;
    if(isPowerOfTwo(n))
        cout << "Power of two" << endl;
    else
        cout << "Not a power of 2" << endl;
    
    return 0;
}
\end{lstlisting}


\subsection{Count set Bits}
\textbf{Code:}

\begin{lstlisting}[language=c++]

int countSetBits(int n) {
    int count = 0;
    while(n>0) {
        if(n&1)
            count++;
        n = (n>>1);
    }
    return count;
}

\end{lstlisting}


Here,\\ \\
\fbox{\begin{minipage}{14 cm}
\begin{align*}
	\text{if } n =&\hspace{4pt} 20\\
	(20)_2 =&\hspace{4pt} 00010100\\
	\text{Operation-1}:&\hspace{4pt} 00001010 \tab n>>1 \\
	\text{Operation-2}:&\hspace{4pt} 00000101 \tab n>>2 \\
	\text{Operation-3}:&\hspace{4pt} 00000010 \tab n>>3 \\
	\text{Operation-4}:&\hspace{4pt} 00000001 \tab n>>4 \\
	\text{Operation-5}:&\hspace{4pt} 00000000 \tab n>>5 \\
\end{align*}
\end{minipage}
}\\

\vspace{1.5 cm}
\textbf{Count Set Bits Hack:}
\begin{lstlisting}[language=c++]

int countSetBits(int n) {
    int count = 0;
    while(n>0) {
        n = n & (n-1);
        count++;
    }
    return count;
}

\end{lstlisting}

Here,\\ \\
\fbox{\begin{minipage}{14 cm}
\begin{align*}
	\text{if } n =&\hspace{4pt} 20\\
	(20)_2 =&\hspace{4pt} 00010100\\
	\text{Operation-1}:&\hspace{4pt} 00010000 \tab n\&(n-1) \\
	\text{Operation-2}:&\hspace{4pt} 00000000 \tab n\&(n-1) \\
\end{align*}
\end{minipage}
}

% ---------------------------------------------
\subsection{Check whether a number is power of 4}
\ding{228} First check whether the number is power of 2 or not\\
\ding{228} Then check for the numbers that are power of 4

\setlength{\columnsep}{3 cm}
\begin{multicols}{2}

\begin{center}
\begin{align*}
	(-16)_2 =&\hspace{4pt} 1111111111110000\\
	(0x5555)_2 =&\hspace{4pt} 0101010101010101\\
	\hline
	AND(\&) =&\hspace{4pt} 0101010101010000 > 0\\	
\end{align*}
\end{center}

\columnbreak

\begin{center}
\begin{align*}
	(16)_2 =&\hspace{4pt} 0000000000010000\\
	(0x5555)_2 =&\hspace{4pt} 0101010101010101\\
	\hline
	AND(\&) =&\hspace{4pt} 0000000000010000 > 0\\
\end{align*}
\end{center}

\end{multicols}

\textbf{Code to check if a number is power of 4}

\begin{lstlisting}[language=c++]

bool isPowerOfFour(int n) {
    bool multipleOfTwo = (n&(n-1)); // Check for power of 2
    bool setAtEven = !(n&(0x55555555)); // Check for power of 4
    
    return (multipleOfTwo or setAtEven) ? false : true;
}
\end{lstlisting}


%%%%%%%%%%%%%%%%%%%%%%%%%%% Bitmanipulation Problems %%%%%%%%%%%%%%%%%%%%%%%%%%%%
\newpage
\section{\textbf{Bit-Manipulation Problems}}

\subsection{Unique Number-1}
Given 2n+1 numbers where every number comes twice expect one number. Find out that unique number.

\begin{center}
	\fbox{
 \begin{minipage}{\linewidth}
  \begin{equation*}
   	\underbrace{2 \oplus 2}_{0}\oplus\underbrace{7}_{7}\oplus\underbrace{3 \oplus 3}_{0}							\oplus\underbrace{4 \oplus 4}_{0}
  \end{equation*}
 \end{minipage}
}
\end{center}

\textbf{Unique Number-1 Code:}
\begin{lstlisting}[language=c++]

 int getUniqueNum(vector <int> arr) {
    int xOR = 0;
    for(auto i:arr) { xOR ^= i; }
    return xOR;
}
\end{lstlisting}

\subsection{Finding all Subsets}
Given a string and find out the all possible subset of that string
\begin{equation*}
	\textsl{Sub sets of } (abc) = \lbrace \phi, a, b, c, ab, ac, bc, abc \rbrace
\end{equation*}

\begin{center}
	\begin{tabular}{| c | c | c | c | c | c | c | c |}
	\hline
	$\phi $ &   a &  b  &  ab & c   & c a & cb  & abc \\ \hline
	000     & 001 & 010 & 011 & 100 & 101 & 110 & 111 \\
	\hline
	\end{tabular}
\end{center}

\textbf{Code Implementation:}
\begin{lstlisting}[language=c++]
void overLayNumber(string str, int n) {
    int j=0;
    while(n) {
        if(n&1) {
            cout << str[j];
        }
        n = (n>>1);
        j++;
    }
    cout << endl;
}

int main () {
    string str;
    cin >> str;
    for(int i=0; i<(1<<str.length()); i++) {
        overLayNumber(str,i);
    }
}

\end{lstlisting}

\subsection{Binary Exponentiation}
Binary exponentiation (also known as exponentiation by squaring) is a trick which allows to calculate $a^n$ using only $O(\log_2{n})$ multiplications (instead of $O(n)$ multiplications required by the naive approach).\\

\textbf{Algorithm :}\\
Raising $a$ to the power of $n$ is expressed naively as multiplication by $a$ done $n-1$ times: $a^{n} = a \cdot a \cdot \ldots \cdot a$. However, this approach is not practical for large $a$ or $n$.\\
The idea of binary exponentiation is, that we split the work using the binary representation of the exponent.

Let's write, $n$ in base 2, for example:\\
	$$3^{13} = 3^{1101_2} = 3^8 \cdot 3^4 \cdot 3^1$$

Since the number $n$ has exactly $\lfloor \log_2 n \rfloor + 1$ digits in base 2, we only need to perform $O(\log n)$ multiplications, if we know the powers $a^1, a^2, a^4, a^8, \dots, a^{2^{\lfloor \log n \rfloor}}$.\\

\textbf{Code :}
\begin{lstlisting}[language=c++]
long long binaryExp(int a, int n) {
    long long result = 1;
    while(n) {
        if(n&1) {
            result *= a;
        }
        a *= a;
        n >>= 1;
    }
    return result;
}
\end{lstlisting}

\subsection{Compute $x^n$ $mod$ $m$}
Since, we know that,
\begin{equation*}
	(a.b.c)\text{ mod }x = \big((a\text{ mod }x).(b\text{ mod }x).(c\text{ mod }x)\big)\text{ mod }x
\end{equation*}

\textbf{Code :}
\begin{lstlisting}[language=c++]
int expMod(int x, int n, int m) {
    int exp = 1;
    while(n) {
        if(n&1) exp = (exp*x) % m;
        x = (x*x) % m;
        n >>= 1;
    }
    return exp;
}
\end{lstlisting}
						  
