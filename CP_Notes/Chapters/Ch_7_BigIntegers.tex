\newpage
\chapter{\textbf{Big Integer}}
In C or C++, we have different types of data types like integers, long, float, characters etc. Each data type occupies some amount of memory. There is a range of numbers that can be occupied by that data type.\\
For example, an integer occupies 4 bytes of memory so that we can have numbers from -2147483648 to +2147463647.\\
So, if we want to have an integer where the number of digits in decimal format is 22 or more, then we cannot store it using primitive data types.\\

%%%%%%%%%%%%%%%%%%%%%%%% Addition %%%%%%%%%%%%%%%%%%%%
\newpage
\section{\textbf{Addition of two Big Integers}}

\setlength{\columnsep}{1 cm}
\begin{multicols}{2}
\begin{lstlisting}[language=c++]
#include <iostream>
#include <algorithm>
#include <string>
using namespace std;

// Char to int .........
int charToInt ( char ch) {
    return ch - '0';
}

// Int to char ........
char intToChar (int digit) {
    return digit + '0';
}

// Addition method ....................
string addition(string n1, string n2) {

    // make sure n1 <= n2 ........
    if(n1.length() > n2.length()) {
        swap(n1,n2);
    }

    // Reverse the numbers .....
    reverse(n1.begin(),n1.end());
    reverse(n2.begin(),n2.end());

    int carry = 0;
    string result;

    for(int i=0; i<n1.length(); i++) {
        int d1 = charToInt(n1[i]);
        int d2 = charToInt(n2[i]);
        int sum = d1 + d2 + carry;
        int outputDigit = sum % 10;
        carry = sum/10;
        result.push_back(intToChar(outputDigit));
    }

    for(int i=n1.length(); i<n2.length(); i++) {
        int d = charToInt(n2[i]);
        int sum = d + carry;
        int outputDigit = sum % 10;
        carry = sum /10;
        result.push_back(intToChar(outputDigit));
    }

    if(carry) {
        result.push_back('1');
    }

    // Reaverse the result ........
    reverse(result.begin(), result.end());

    return result;
}

int main () {
    string s1,s2;
    cin >> s1 >> s2;

    string sum = addition(s1,s2);
    cout << sum << endl;

    return 0;
}
\end{lstlisting}

\end{multicols}


%%%%%%%%%%%%%%%%%%% Big Factorial %%%%%%%%%%%%%%%%
\newpage
\section{Big Factorial}

\setlength{\columnsep}{1 cm}
\begin{multicols}{2}

\begin{lstlisting}[language=C++]
#include <iostream>
#include <algorithm>
#include <string>
using namespace std;

// ....... Char to Int ...........
int charToInt (char ch) {
    return ch - '0';
}

// ....... Int to Char ..........
char intToChar (int n) {
    return n + '0';
}

// ......... Multiplication ............
string multiplication(string n1, int n2) {

    reverse(n1.begin(), n1.end());
    string result;
    int carry = 0;
    
    for (char ch: n1) {
        int mul = charToInt(ch)*n2 + carry;
        result.push_back(intToChar(mul%10));
        carry = mul/10;
    }

    while(carry) {
        result.push_back(intToChar(carry%10));
        carry /= 10;
    }

    reverse(result.begin(), result.end());
    return result;
}

// ......... Big Factorial Method ..........
string bigFactorial(int n) {
    string fact = "1";
    for(int i=2; i<=n; i++) {
        fact = multiplication(fact,i);
    }
    return fact;
}
\end{lstlisting}

\columnbreak

\begin{lstlisting}[language=C++]
int main () {
    int n;
    cin >> n;
    cout << bigFactorial(n) << endl;
    return 0;
}
\end{lstlisting}

\end{multicols}


