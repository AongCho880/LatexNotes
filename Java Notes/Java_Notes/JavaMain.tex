\documentclass[14pt]{report}

\usepackage[utf8]{inputenc} % Enables LaTeX to handle Unicode Characters directly 
\usepackage{mathptmx} % Used to change the default font in math mode
\usepackage{amsmath} % To write mathematical equation
\usepackage{graphicx} % Used to include external graphics
\usepackage{animate} % for animation
\usepackage{subcaption} % To present multiple related images or tables within a single figure
\usepackage{booktabs} % to creat high quality tables
\usepackage{circuitikz} % to draw electrical circuit
\usepackage{siunitx} % for si unit
\usepackage{tikz} % to draaw flow chart
\usetikzlibrary{shapes.geometric, arrows} % for flow chart
\usepackage{float} % enhanced control over the placement of floating element
\usepackage{enumerate} % to customize the appereance of lists
\usepackage[colorlinks,linkcolor=black, citecolor=blue, urlcolor=blue]{hyperref} % to enable the creation of hyperlinks
\usepackage{geometry} % for margin specification
\geometry{left=3.5cm,right=2.5cm,top=2.5cm,bottom=2.5cm} % set margin
\usepackage{setspace} % set space between line
\onehalfspacing
% \usepackage{fontspec} % to utilize fonts installed on system directly
% \setmainfont{Times New Roman}
\usepackage{titlesec}
\usepackage{fancyhdr} % to customize header and footer
\usepackage{lipsum} % to generate dummy text
\usepackage{lastpage} % page numbering
\usepackage{etoolbox} % provides tools for modifying and patching LaTeX commands. 
\usepackage{cite} % of citations and bibliographies in LaTeX.

\usepackage{lmodern} % to control font-size
%% {\fontsize{font size}{base line strech} \selectfont}
%% {\fontsize{40}{48} \selectfont Dynamic} % syntex to control fontsize.


%%%%%%%%%%%%%%%%%%%%%%% Coding Style %%%%%%%%%%%%%%%%%%%%%%%%%
\usepackage{listings}
\usepackage{xcolor}

%New colors defined below
\definecolor{codegreen}{rgb}{0,0.6,0}
\definecolor{codegray}{rgb}{0.5,0.5,0.5}
\definecolor{codepurple}{rgb}{0.58,0,0.82}
\definecolor{backcolour}{rgb}{0.95,0.95,0.92}

%Code listing style named "mystyle"
\lstdefinestyle{mystyle}{
  backgroundcolor=\color{backcolour}, commentstyle=\color{codegreen},
  keywordstyle=\color{magenta},
  numberstyle=\tiny\color{codegray},
  stringstyle=\color{codepurple},
  basicstyle=\ttfamily\small,
  breakatwhitespace=false,         
  breaklines=true,                 
  captionpos=b,                    
  keepspaces=true,                 
  numbers=left,                    
  numbersep=5pt,                  
  showspaces=false,                
  showstringspaces=false,
  showtabs=false,                  
  tabsize=2
}

\lstset{style=mystyle}

%%%%%%%%%%%%%%%%%%%%%%%%%%%%%%%%%%%
\usepackage{tabularx}
\usepackage{makecell}
\usepackage{multirow}
\usepackage{multicol}
\usepackage{hhline}
\usepackage{pgf-pie}
\usetikzlibrary{backgrounds}
\usepackage{courier}
\newcommand\tab[1][1cm]{\hspace*{#1}}

%%%%%% To insert PDF cover page %%%%%%%
\usepackage{titlepic}
\usepackage{pdfpages}

\title{My Knowledge On Java}
\author{Aong Cho Marma}
\date{13 July, 2023}

\begin{document}
	\includepdf[pages=1, fitpaper]{CoverPage/CoverPage}
	
	\maketitle
	
	%%%%%%%%%%%%%%%%%%% Table of Contents %%%%%%%%%%%%%%%%
	\tableofcontents
	\pagenumbering{\roman{i}}	
	
	%%%%%%%%%%%%%%%%%% Page Numbering Style %%%%%%%%%%%%%%
	\pagenumbering{arabic}
	\patchcmd{\chapter}{\thispagestyle{plain}}{\thispagestyle{fancy}}{}{}
	\fancypagestyle{main}{
    	\fancyhf{} % clear all header and footer fields
    	\renewcommand{\headrulewidth}{0pt} % remove header rule
    	\renewcommand{\footrulewidth}{1pt} % set footerr rule
    	\fancyfoot[L]{
    		\href{mailto:aongcho880@gmail.com}{\underline{Aong Cho Marma}}\\
    		\fontsize{11}{13}\selectfont \leftmark
    	} % chapter name at left footer
    	\fancyfoot[R]{Page \thepage\ of \pageref{LastPage}} % page number at center footer
	}
	
	\pagestyle{main} % set the main page style
	\setcounter{page}{1}
	% Define style for equation numbers
	\renewcommand{\theequation}{\arabic{chapter}.\arabic{equation}}
	
	%%%%%%%%%%%%%%%%%%%%%%%%%%%%%%%%%%%%%%%%%%%%%
	
	%%%%%%%%%%%%%%% Chapters %%%%%%%%%%%%%%%%%%%%
	\newpage
\chapter{INTRODUCTION}
Java is a General Purpose Programming Language (GPL) and also powerful programming language.
Java developed at \textbf{Sun Microsystems} which was purchased by \textbf{Oracle} in 2010.
Java is GPL because it is used to solve a wide variety of problems and build software.

%%%%%%%%%%%%%%%%%%%%%%%%%%% Specification %%%%%%%%%%%%%%%%%%%%%%%%%%%%%%%
\section{JAVA LANGUAGE SPECIFICATION}

%%%%%%%%%%%%%%%%%%%%%%%%%%%%%%%%%%%%%
\subsection{THE SYNTAX \& SEMANTICS}
To write English we should follow some rules (Grammar). Also, to write java we should follow some rules that is called \textbf{syntax \& semantics}\\
\textbf{Example:}
\begin{itemize}
	\item[{\LARGE $\diamond$}] He \textcolor{red}{are} playing $\Rightarrow$ Syntax error.(Grammar)
	\item[{\LARGE $\diamond$}] He is hello and bye $\Rightarrow$ Semantic error.(Meaning)
\end{itemize}


%%%%%%%%%%%%%%%%%%%%%%%%%%%%%%%%%%%%%%%
\subsection{API}
\textbf{Application Programming Interface (API)} also known as \textbf{library}. It contains predefined Java code that we can use to develop Java programs. It makes faster and easier development process. Because we do not need to write everything from scratch.

%%%%%%%%%%%%%%%%%%%%%%%%%%%%%%%%%%%%%%%
\subsection{EDITIONS OF JAVA}
Java comes in three editions
\begin{itemize}
	\item[{\LARGE $\diamond$}] \textbf{Standard Edition(SE):} Develop applications that run on desktop.
	\item[{\LARGE $\diamond$}] \textbf{Enterprise Edition(EE):} Develop server-side applications.
	\item[{\LARGE $\diamond$}] \textbf{Micro Edition(ME):} Develop applications for mobile devices.
\end{itemize}

\begin{quotation}
	\textbf{NOTE:} Java SE is the foundation of all other editions.
\end{quotation}


%%%%%%%%%%%%%%%%%%%%%%%%%
\subsection{JDK}
Java Development Kit (JDK)
\begin{itemize}
	\item[{\LARGE $\diamond$}] Set of programs that enable us to develop our programs.
	
	\item[{\LARGE $\diamond$}] Contains \textbf{JRE(Java Runtime Environment)} that is used to run out programs.
	
	\item[{\LARGE $\diamond$}] \textbf{JDK \& JRE} contain \textbf{JVM (Java Virtual Machine)}.
	
	\item[{\LARGE $\diamond$}] \textbf{JVM} executes our java programs on different machines that makes Java independent.
\end{itemize}

%%%%%%%%%%%%%%%%%%%%%%%%%%%%%%%%
\subsection{IDE}
\textbf{Integrated Development Environment (IDE)} is a program that allows us to-
\begin{itemize}
	\item[{\LARGE $\diamond$}] \textbf{Write}: Write source code
	\item[{\LARGE $\diamond$}] \textbf{Compile}: Translate source code to machine code
	\item[{\LARGE $\diamond$}] \textbf{Debug}: Tools to find errors
	\item[{\LARGE $\diamond$}] \textbf{Build}: Files that can be executed by JVM
	\item[{\LARGE $\diamond$}] \textbf{Run}: Execute program
\end{itemize}
IDE makes development faster and easier. NetBeans, Eclipse, IntelliJ IDE are the popular Java IDEs.
\begin{quotation}
	\textbf{NOTE:} The Java source code first compiled into a binary byte code using Java compiler, then this byte code runs on the JVM, Which is a software based interpreter. So Java is considered as both interpreted and compiled.
\end{quotation} 


%%%%%%%%%%%%%%%%%%%%%%% Anatomy of Java Program %%%%%%%%%%%%%%%
\section{ANATOMY OF JAVA PROGRAM}

%%%%%%%%%%%%%
\subsection{CLASS}
A blueprint to create \textsl{OBJECTS}.

%%%%%%
\subsubsection{CLASS STRUCTURE}
\begin{lstlisting}[language=java]
	class class_name {
		code block
	}
	// "class" is keyword.
\end{lstlisting}

%%%%%%%%%%%%%%%
\subsection{OBJECTS}
An instance of a \textsl{CLASS}.

%%%%%%%%%%%%%%%
\subsection{METHOD}
Group of instruction to do a specific task.

\subsubsection{METHOD STRUCTURE}
Each method consists of 4 main parts.
\begin{itemize}
	\item[{\LARGE $\diamond$}] Return Type
	\item[{\LARGE $\diamond$}] Method Name
	\item[{\LARGE $\diamond$}]Parameter
	\item[{\LARGE $\diamond$}]Code Block
\end{itemize}

\begin{lstlisting}[language=java]
	return_type method_name(parameter) {
		code block
	}
\end{lstlisting}

\begin{quotation}
	\textbf{NOTE :} Every method is written inside a \textsl{CLASS}.
\end{quotation}

\subsubsection{CALLING A METHOD}
It is basically using the method
\begin{lstlisting}[language=java]
	method_name(parameter);
\end{lstlisting}

\begin{quotation}
	\textbf{NOTE :} The \textsl{main()} method is automatically called when we run the JAVA program.
	\begin{itemize}
		\item It is the first method that is called.
		\item It is the starting point of execution of a program.
	\end{itemize}
\end{quotation}

%%%%%%%%%%%%%%%%%%%%%%%%%%%%
\subsection{ACCESS MODIFIERS}
The access modifiers in JAVA specifies the accessibility of a field, method, constructor, or class.\\
There are four types of JAVA access modifiers :
\begin{itemize}
	\item[{\LARGE $\diamond$}] \textbf{Private :} Access level only within the class, can not be accessed from outside the class.
	
	\item[{\LARGE $\diamond$}] \textbf{Default :} Access level only within the package. If do not specify any access leve then it will be the default.
	
	\item[{\LARGE $\diamond$}] \textbf{Protected :} Access level within the package and outside the package \textsl{through child class}. If do not make the child class, it can not be accessed from outside the package.
	
	\item[{\LARGE $\diamond$}] \textbf{Public :} Access level everywhere, within or outside the class and package.
\end{itemize}


%%%%%%%%%%%%%%%%%%%%%%%%%%%
\subsection{NAMING CONVENTIONS}
How to write name in programming.
\begin{itemize}
	\item[{\LARGE $\diamond$}] \textbf{Pascal Case Convention :}
	\begin{itemize}
		\item ThisIsAName
		\item Naming \textsl{Class}
	\end{itemize}
		
		
	\item[{\LARGE $\diamond$}] \textbf{Camel Case Convention :}
	\begin{itemize}
		\item thisIsAName
		\item Naming \textsl{Methods \& Variables}
	\end{itemize}
	
	\item[{\LARGE $\diamond$}] \textbf{Snake Case Convention :}
	\begin{itemize}
		\item this\_is\_name
	\end{itemize}
	
\end{itemize}


%%%%%%%%%%%%%%%%%%%%%%%%%%%%
\subsection{JAVA PROGRAM STRUCTURE}
\begin{lstlisting}[language=java]
	public class Main {
		public static void main ( String[] args) {
			code_block
		}
	}
\end{lstlisting}

%%%%%%%%%%%%%%%%%
\subsection{PACKAGE}
A container for Classes.

\begin{quotation}
	\textbf{NOTE :}
	\begin{itemize}
		\item Package contains Classes
		\item Class contains Methods \&
		\item Method contains code blocks
	\end{itemize}
\end{quotation}
	\chapter{File Handling In Java}

% ---------------- Section 1 -----------------------
\section{Create, Delete and Get The Path Of The Dir.}
\lstset{
  basicstyle=\fontsize{8}{10}\selectfont\ttfamily,
}
\begin{lstlisting}[language=java]
package fileHandling;

import java.io.File;

public class A_CreateDir {

	public static void main(String[] args) {
		// It will create the directory at the current project directory.
		File dir = new File("Test"); // we can use directory path also
		dir.mkdir();
		
		//---- Get The File Directory
		String dirPath = dir.getAbsolutePath();
		String name = dir.getName();
		System.out.println("Dir Name: "+name+"\nPath: "+dirPath);
		
		//---- Delete Directory If Exist
		if(dir.delete()) {
			System.out.println(name+" directory has been deleted.");
		}
		
	}
}

\end{lstlisting}


% ---------------- Section 2 -----------------------
\newpage
\section{Create File and Delete File}
\lstset{
  	basicstyle=\fontsize{8}{10}\selectfont\ttfamily,
}
\begin{lstlisting}[language=java]
package fileHandling;

import java.io.File;
import java.util.*;

public class B_CreateFile {

	public static void main(String[] args) {
		//---- File Must Be Exist Other Wise Create Directory
		//---- Set The Directory path
		File dir = new File("TextFile");
		
		//------ Get The Directory Path
		String path = dir.getAbsolutePath();
		
		//---- Set File Name And Path
		File file1 = new File(path+"/file1.txt");
		File file2 = new File(path+"/file2.txt");
		
		try {
			
			// ---- Create New File
			file1.createNewFile();
			file2.createNewFile();
			
		} catch (Exception e) {
			System.out.println(e);
		}
		
		//--- Check If A File Exist
		if(file2.exists()) {
			//---- Delete File
			file2.delete();
		}
	}
}

\end{lstlisting}


%------------- Ways to write data into a file -----------------
\newpage
\section{Ways To Write Data Into a File in Java}

\subsection{FileWriter Class}
\begin{itemize}
	\item \textit{FileWriter} is used for writing character data into a file.
	\item It`s suitable for writing simple text-based data.
	\item You can write strings and characters directly to the file.
\end{itemize}
\textbf{Code:}
\lstset{
  	basicstyle=\fontsize{8}{10}\selectfont\ttfamily,
}
\begin{lstlisting}[language=java]
package fileHandling;

import java.io.File;
import java.io.FileWriter;
import java.io.IOException;
import java.util.Formatter;

public class WriteInFile {
	public static void main(String[] args) {	
		FileWriter writer = null;
		
		try {
			writer = new FileWriter("Output.txt");
			
			char c = 'A';
			writer.write(c); //----- write single character
			writer.write('\n');
			
			char[] charArry = {'X', 'Y', 'Z'};
			writer.write(charArry); //--- Write array of characters
			writer.write('\n');
			
			String str = "This is a string";
			writer.write(str);	//--- write string
			writer.write('\n');
			
			writer.flush(); //- Ensure the data is immediately written to the file
			writer.close();
		} catch (IOException e) {
			System.out.println(e);
		}
	}
}
\end{lstlisting}

%--------------------------------
\newpage
\subsection{BufferedWriter Class}
\begin{itemize}
	\item This approach combines a \textbf{BufferedWriter} with a \textbf{FileWriter} to improve efficiency when writing large amounts of text data.
	\item It reduces the number of disk writes and is useful for optimizing performance
\end{itemize}
\textbf{Code:}
\lstset{
  	basicstyle=\fontsize{8}{10}\selectfont\ttfamily,
}
\begin{lstlisting}[language=java]
package fileHandling;

import java.io.BufferedWriter;
import java.io.FileWriter;
import java.io.IOException;

public class WriteInFile {

	public static void main(String[] args) {
		
		FileWriter fWriter = null;
		BufferedWriter bWriter = null;
		
		try {
			fWriter = new FileWriter("Output.txt");
			bWriter = new BufferedWriter(fWriter);
			
			bWriter.write('A');
			bWriter.newLine();
			bWriter.write("This is a String");
			bWriter.flush();
			
			fWriter.close();
			bWriter.close();
		} catch (IOException e) {
			System.out.println(e);
		}
	}

}

\end{lstlisting}


%--------------------------------
\newpage
\subsection{PrintWriter Class}
\begin{itemize}
	\item \textbf{PrintWriter} is useful for writing formatted text data into a file.
	\item It provides methods like \textbf{\texttt{printf}} and \textbf{\texttt{println}} for formatting output.
\end{itemize}

\textbf{Code:}
\lstset{
  	basicstyle=\fontsize{8}{10}\selectfont\ttfamily,
}
\begin{lstlisting}[language=java]
package fileHandling;

import java.io.FileWriter;
import java.io.IOException;
import java.io.PrintWriter;

public class WriteInFile {

	public static void main(String[] args) {
		
		FileWriter fWriter = null;
		PrintWriter pWriter = null;
		
		try {
			fWriter = new FileWriter("Output.txt");
			pWriter = new PrintWriter(fWriter);
			

			pWriter.println("Hello World");
			pWriter.println("This is PrintWriter");
			pWriter.printf("Formatted Output: %s : %d + %d = %d %n", "Sum",5,7,12);
			pWriter.write("New Line");
			pWriter.flush();
			
			fWriter.close();
			pWriter.close();
		} catch (IOException e) {
			System.out.println(e);
		}
	}

}

\end{lstlisting}


%-------------- Java Formatter Class ----------------
\newpage
\section{Java Formatter Class}
The Java \textit{Formatter} class is defined in the \textit{java.util} package and is declared final. It, therefore, cannot be extended or sub-classed.\\
With the help of this class, we can send formatted outputs to other outputs streams or devices, such as a GUI component or to a file apart from standard output.\\
\textbf{Formatter Construction}
\begin{itemize}
	\item \textbf{Formatter() :}
	\item \textbf{Formatter(Appendable a) :}
	\item \textbf{Formatter(Appendable a, Locale loc) :}
	\item \textbf{Formatter(File file) :} The file parameter of this constructor designates a reference to a open file where the output will be streamed.\\
\end{itemize}

\textbf{Using Formatter}
\begin{itemize}
	\item \textbf{\%S or \%s :} Specifies String
	\item \textbf{\%X or \%x :} Specifies hexadecimal integer
	\item \textbf{\%o :} Specifies Octal integer
	\item \textbf{\%d :} Specifies Decimal integer
	\item \textbf{\%c :} Specifies character
	\item \textbf{\%T or \%t :} Specifies Time and Date
	\item \textbf{\%n :} Insets newline character
	\item \textbf{\%B or \%b :} Specifies Boolean
	\item \textbf{\%A or \%a :} Specifies floating point hexadecimal
	\item \textbf{\%f :} Specifies Decimal floating point\\
\end{itemize}

\newpage
\subsection{A Few Quick Examples}

\textbf{Using argument\_index}

\lstset{
  	basicstyle=\fontsize{8}{10}\selectfont\ttfamily,
}
\begin{lstlisting}[language=java]
Formatter f2 = new Formatter();
f2.format("%2$s %1$s %3$s", "fear", "weakness","strengthen");
System.out.println(f2);
f2.close();

// Output: weakness fear strengthen		
\end{lstlisting}


\textbf{Regionalize Date}

\lstset{
  	basicstyle=\fontsize{8}{10}\selectfont\ttfamily,
}
\begin{lstlisting}[language=java]
Formatter f3=new Formatter();
f3.format(Locale.FRENCH,"%1$te %1$tB, %1$tY", Calendar.getInstance());
System.out.println(f3);
f3.close();

Formatter f4=new Formatter();
f4.format(Locale.ENGLISH,"%1$te %1$tB, %1$tY",Calendar.getInstance());
System.out.println(f4);
f4.close();

// Output: 7 octobre, 2023
//  		   7 October, 2023		
\end{lstlisting}


\textbf{Using \%n and \%\% Specifiers}

\lstset{
  	basicstyle=\fontsize{8}{10}\selectfont\ttfamily,
}
\begin{lstlisting}[language=java]
Formatter f = new Formatter();
f.format("Format%n %.2f%% complete", 46.6);
System.out.println(f);
f.close();
	
// Output: Format
//			46.60% complete
\end{lstlisting}

\textbf{Write in a file}

\lstset{
  	basicstyle=\fontsize{8}{10}\selectfont\ttfamily,
}
\begin{lstlisting}[language=java]
public class WriteInFile {
	public static void main(String[] args) {
		//--- "TestFile" directory must be exist in the project directory
		File file = new File("TextFile");
		String path = file.getAbsolutePath();
		System.out.println(path);
		
		try {
			Formatter formatter = new Formatter(path+"/file1.txt");
			formatter.format("%s %s %s\n", "21701002","Aong Cho","CSE");
			formatter.format("%s %s %s\n", "21701001","Taqi Ismile","CSE");
			formatter.close();
		} catch (Exception e) {
			System.out.println(e);
		}
	}
}
\end{lstlisting}






	
	
\end{document}