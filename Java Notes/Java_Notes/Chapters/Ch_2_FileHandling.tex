\chapter{File Handling In Java}

% ---------------- Section 1 -----------------------
\section{Create, Delete and Get The Path Of The Dir.}
\lstset{
  basicstyle=\fontsize{8}{10}\selectfont\ttfamily,
}
\begin{lstlisting}[language=java]
package fileHandling;

import java.io.File;

public class A_CreateDir {

	public static void main(String[] args) {
		// It will create the directory at the current project directory.
		File dir = new File("Test"); // we can use directory path also
		dir.mkdir();
		
		//---- Get The File Directory
		String dirPath = dir.getAbsolutePath();
		String name = dir.getName();
		System.out.println("Dir Name: "+name+"\nPath: "+dirPath);
		
		//---- Delete Directory If Exist
		if(dir.delete()) {
			System.out.println(name+" directory has been deleted.");
		}
		
	}
}

\end{lstlisting}


% ---------------- Section 2 -----------------------
\newpage
\section{Create File and Delete File}
\lstset{
  	basicstyle=\fontsize{8}{10}\selectfont\ttfamily,
}
\begin{lstlisting}[language=java]
package fileHandling;

import java.io.File;
import java.util.*;

public class B_CreateFile {

	public static void main(String[] args) {
		//---- File Must Be Exist Other Wise Create Directory
		//---- Set The Directory path
		File dir = new File("TextFile");
		
		//------ Get The Directory Path
		String path = dir.getAbsolutePath();
		
		//---- Set File Name And Path
		File file1 = new File(path+"/file1.txt");
		File file2 = new File(path+"/file2.txt");
		
		try {
			
			// ---- Create New File
			file1.createNewFile();
			file2.createNewFile();
			
		} catch (Exception e) {
			System.out.println(e);
		}
		
		//--- Check If A File Exist
		if(file2.exists()) {
			//---- Delete File
			file2.delete();
		}
	}
}

\end{lstlisting}


%------------- Ways to write data into a file -----------------
\newpage
\section{Ways To Write Data Into a File in Java}

\subsection{FileWriter Class}
\begin{itemize}
	\item \textit{FileWriter} is used for writing character data into a file.
	\item It`s suitable for writing simple text-based data.
	\item You can write strings and characters directly to the file.
\end{itemize}
\textbf{Code:}
\lstset{
  	basicstyle=\fontsize{8}{10}\selectfont\ttfamily,
}
\begin{lstlisting}[language=java]
package fileHandling;

import java.io.File;
import java.io.FileWriter;
import java.io.IOException;
import java.util.Formatter;

public class WriteInFile {
	public static void main(String[] args) {	
		FileWriter writer = null;
		
		try {
			writer = new FileWriter("Output.txt");
			
			char c = 'A';
			writer.write(c); //----- write single character
			writer.write('\n');
			
			char[] charArry = {'X', 'Y', 'Z'};
			writer.write(charArry); //--- Write array of characters
			writer.write('\n');
			
			String str = "This is a string";
			writer.write(str);	//--- write string
			writer.write('\n');
			
			writer.flush(); //- Ensure the data is immediately written to the file
			writer.close();
		} catch (IOException e) {
			System.out.println(e);
		}
	}
}
\end{lstlisting}

%--------------------------------
\newpage
\subsection{BufferedWriter Class}
\begin{itemize}
	\item This approach combines a \textbf{BufferedWriter} with a \textbf{FileWriter} to improve efficiency when writing large amounts of text data.
	\item It reduces the number of disk writes and is useful for optimizing performance
\end{itemize}
\textbf{Code:}
\lstset{
  	basicstyle=\fontsize{8}{10}\selectfont\ttfamily,
}
\begin{lstlisting}[language=java]
package fileHandling;

import java.io.BufferedWriter;
import java.io.FileWriter;
import java.io.IOException;

public class WriteInFile {

	public static void main(String[] args) {
		
		FileWriter fWriter = null;
		BufferedWriter bWriter = null;
		
		try {
			fWriter = new FileWriter("Output.txt");
			bWriter = new BufferedWriter(fWriter);
			
			bWriter.write('A');
			bWriter.newLine();
			bWriter.write("This is a String");
			bWriter.flush();
			
			fWriter.close();
			bWriter.close();
		} catch (IOException e) {
			System.out.println(e);
		}
	}

}

\end{lstlisting}


%--------------------------------
\newpage
\subsection{PrintWriter Class}
\begin{itemize}
	\item \textbf{PrintWriter} is useful for writing formatted text data into a file.
	\item It provides methods like \textbf{\texttt{printf}} and \textbf{\texttt{println}} for formatting output.
\end{itemize}

\textbf{Code:}
\lstset{
  	basicstyle=\fontsize{8}{10}\selectfont\ttfamily,
}
\begin{lstlisting}[language=java]
package fileHandling;

import java.io.FileWriter;
import java.io.IOException;
import java.io.PrintWriter;

public class WriteInFile {

	public static void main(String[] args) {
		
		FileWriter fWriter = null;
		PrintWriter pWriter = null;
		
		try {
			fWriter = new FileWriter("Output.txt");
			pWriter = new PrintWriter(fWriter);
			

			pWriter.println("Hello World");
			pWriter.println("This is PrintWriter");
			pWriter.printf("Formatted Output: %s : %d + %d = %d %n", "Sum",5,7,12);
			pWriter.write("New Line");
			pWriter.flush();
			
			fWriter.close();
			pWriter.close();
		} catch (IOException e) {
			System.out.println(e);
		}
	}

}

\end{lstlisting}


%-------------- Java Formatter Class ----------------
\newpage
\section{Java Formatter Class}
The Java \textit{Formatter} class is defined in the \textit{java.util} package and is declared final. It, therefore, cannot be extended or sub-classed.\\
With the help of this class, we can send formatted outputs to other outputs streams or devices, such as a GUI component or to a file apart from standard output.\\
\textbf{Formatter Construction}
\begin{itemize}
	\item \textbf{Formatter() :}
	\item \textbf{Formatter(Appendable a) :}
	\item \textbf{Formatter(Appendable a, Locale loc) :}
	\item \textbf{Formatter(File file) :} The file parameter of this constructor designates a reference to a open file where the output will be streamed.\\
\end{itemize}

\textbf{Using Formatter}
\begin{itemize}
	\item \textbf{\%S or \%s :} Specifies String
	\item \textbf{\%X or \%x :} Specifies hexadecimal integer
	\item \textbf{\%o :} Specifies Octal integer
	\item \textbf{\%d :} Specifies Decimal integer
	\item \textbf{\%c :} Specifies character
	\item \textbf{\%T or \%t :} Specifies Time and Date
	\item \textbf{\%n :} Insets newline character
	\item \textbf{\%B or \%b :} Specifies Boolean
	\item \textbf{\%A or \%a :} Specifies floating point hexadecimal
	\item \textbf{\%f :} Specifies Decimal floating point\\
\end{itemize}

\newpage
\subsection{A Few Quick Examples}

\textbf{Using argument\_index}

\lstset{
  	basicstyle=\fontsize{8}{10}\selectfont\ttfamily,
}
\begin{lstlisting}[language=java]
Formatter f2 = new Formatter();
f2.format("%2$s %1$s %3$s", "fear", "weakness","strengthen");
System.out.println(f2);
f2.close();

// Output: weakness fear strengthen		
\end{lstlisting}


\textbf{Regionalize Date}

\lstset{
  	basicstyle=\fontsize{8}{10}\selectfont\ttfamily,
}
\begin{lstlisting}[language=java]
Formatter f3=new Formatter();
f3.format(Locale.FRENCH,"%1$te %1$tB, %1$tY", Calendar.getInstance());
System.out.println(f3);
f3.close();

Formatter f4=new Formatter();
f4.format(Locale.ENGLISH,"%1$te %1$tB, %1$tY",Calendar.getInstance());
System.out.println(f4);
f4.close();

// Output: 7 octobre, 2023
//  		   7 October, 2023		
\end{lstlisting}


\textbf{Using \%n and \%\% Specifiers}

\lstset{
  	basicstyle=\fontsize{8}{10}\selectfont\ttfamily,
}
\begin{lstlisting}[language=java]
Formatter f = new Formatter();
f.format("Format%n %.2f%% complete", 46.6);
System.out.println(f);
f.close();
	
// Output: Format
//			46.60% complete
\end{lstlisting}

\textbf{Write in a file}

\lstset{
  	basicstyle=\fontsize{8}{10}\selectfont\ttfamily,
}
\begin{lstlisting}[language=java]
public class WriteInFile {
	public static void main(String[] args) {
		//--- "TestFile" directory must be exist in the project directory
		File file = new File("TextFile");
		String path = file.getAbsolutePath();
		System.out.println(path);
		
		try {
			Formatter formatter = new Formatter(path+"/file1.txt");
			formatter.format("%s %s %s\n", "21701002","Aong Cho","CSE");
			formatter.format("%s %s %s\n", "21701001","Taqi Ismile","CSE");
			formatter.close();
		} catch (Exception e) {
			System.out.println(e);
		}
	}
}
\end{lstlisting}





