\documentclass[a4paper,12pt]{book}
\usepackage[T1]{fontenc}
\usepackage{titlesec}

\titleformat
{\chapter} % command
[display] % shape
{\bfseries\Large\itshape} % format
{\ \thechapter} % label
{0.5ex} % sep
{
    \rule{\textwidth}{1pt}
    \vspace{1ex}
    \centering
} % before-code
[
\vspace{-0.5ex}%
\rule{\textwidth}{0.3pt}
] % after-code


\titleformat{\section}[wrap]
{\normalfont\bfseries}
{\thesection.}{0.5em}{}

\titlespacing{\section}{12pc}{1.5ex plus .1ex minus .2ex}{1pc}

\begin{document}
\chapter{The Eighth Five Year Plan(July
2020 – June 2025)\\
Addressing COVID-19 Challenges and
Sustainable LDC Graduation}
\section{Introduction}
\begin{enumerate}
   \item The Eighth Five Year Plan (8
th FYP),which will be implemented during FY2021-25, is
currently being finalised before placing it to the NEC for final approval
   
   \begin{enumerate}
     \item The draft plan document, which was prepared early, had to be revisited on account of adjustment because of the COVID-19 pandemic   
       \end{enumerate}
       \item The 8 th FYP is expected to take into cognisance the experiences and lessons learnt from the 7 th FYP (FY2016–20)
\begin{enumerate}
     \item The experience and result of implementing the 7 th FYP have been rather mixed. The COVID-19 pandemic has caused major disruptions in macroeconomic management as the plan period reached its final year inFY2020   
       \end{enumerate}
       \item The 8 th FYP will need to be informed by three key challenges:

      \begin{enumerate}
     \item Economic recovery and rebound in post-COVID period
     \item Graduation from the LDC group by 2024
     \item Second crucial (five-year) lap in implementing the SDGs by 2030
       \end{enumerate}
       \item 8th FYP must also address electoral pledges of the ruling party made prior to 2018 National Election
     \end{enumerate}
     
   \section{Two broad themes of the plan:}
   \textbf{Promoting Prosperity:} \\
The plan has emphasized on appropriate policies,
frameworks and devised suitable and sustainable development strategies for promoting prosperity. For this, the first step is to bring Bangladesh closer to attaining Upper Middle-Income Country (UMIC) status, major Sustainable Development Goal (SDG) targets, and eliminating extreme poverty.\\
\textbf{Fostering Inclusivity:} A broad-based strategy of inclusiveness with a view to empowering every citizen to participate fully and benefit from the development process and helping the poor and vulnerable with social protection- based income transfers has been adopted in the plan.


\section{Plan is divided into two main parts:}
\textbf{Macroeconomic perspective:}The first part delineates the
macroeconomic framework for the plan period (July 2020-June 2025) along
with strategic directions and policy framework for promoting inclusiveness, reducing poverty and inequality. It also describes the resource envelop and overall fiscal management tools of the government and specifies the Development Results Framework (DRF) for proper monitoring and evaluation.\\
\textbf{Sectoral Strategies:} The second part sets out the sectoral strategies for thirteen sectors (except defense) with some specific targets to attain by FY 2025. The ministries/divisions are expected to follow these sectoral strategies and action measures while preparing their sector specific projects and programs to achieve their respective targets set in the 8 th Five Year Plan.

\section{The 8th Plan centers on six sub-core themes:}
\begin{enumerate}
    \item \textbf{Rapid recovery from \textbf{COVID-19}} to restore human health, confidence, employment, income and economic
activities;
\item  A broad-based \textbf{strategy of inclusiveness} with a view to empowering every citizen to participate fully and
benefit from the development process and helping the poor and vulnerable with social protection- based income
transfers;
\item \textbf{\textbf{GDP} growth acceleration}, employment generation, productivity acceleration and rapid poverty reduction;
\item A \textbf{sustainable development pathway} that is resilient to disaster and climate change, entails sustainable use
of natural resources; and successfully managing the inevitable urbanization transition;
\item Development and improvement of critical institutions necessary to \textbf{lead the economy to \textbf{UMIC} status};
\item \textbf{Attaining \textbf{SDG} targets} and coping up the impact of Least Developed Country (LDC) graduation
(sustainable transition with exploring alternatives to the erosion of preferential benefits and not to go back to
previous LDC status).
\end{enumerate}

\textbf{\textbf{\textbf{\textbf{Here add this table}}}}

\section{Challenges in implementations of the plan:-}
During the implementation period of the 8FYP, the government will face a number of challenges. The \textbf{four} specific ones are the following:
\begin{enumerate}
    \item Covid-19 pandemic
\item Graduation from the least developed country (LDC) category
\item Implementation of the Sustainable Development Goals (SDGs) and
\item Climate change vulnerability.
\end{enumerate} 
Besides, Russia Ukraine war, rise in fuel price, disruption in supply chain and currency devaluation are some other major challenges that government needs to confront with.

\section{Focus of the Plan:}
\textbf{The plan mainly focuses on 7 areas:}
\begin{enumerate}
    \item Labour-intensive manufacturing
\item Export-oriented manufacturing-led growth
\item Agricultural diversification
\item Dynamism in cottage, small and medium enterprises
\item Modern services sector
\item ICT based entrepreneurship, and
\item Overseas employment.
\end{enumerate}

\section{8FYP targets for indicators:}
\textbf{\textbf{\textbf{here include second picture item with special itemize sign}}}

\section{Key Takeaways:-}
\begin{enumerate}
    \item Bangladesh has targeted 8.51 per cent economic growth for the 8th Five Year Plan. To accommodate this growth target, the gross investment needs to be raised to 36.59 per cent of GDP by FY 2025.
    \item As per the Planning Minister, some Tk. 64,959.8 billion is required to implement the 8th Five-Year Plan. Out of the amount, 94.9\% will be mobilized from the domestic source, while 5.1\% will be financed from external resources.
\item Since the beginning of the COVID-19 pandemic, the government has taken various steps to combat its fallout. It has taken a comprehensive plan to overcome the negative impacts of pandemic on economy and people.
\item The comprehensive plan is based on 4 (four) main strategies --discouraging luxury expenditures and prioritizing government spending that creates job, creating loan facilities through commercial banks at subsidized interest rate for the affected industries and businesses, expanding the coverage of the
government’s social safety net programs.
\item In light of the comprehensive plan and strategies, the government has declared a number of stimulus packages to support the emergency healthcare services to protect jobs and achieve smooth economic recovery.
\end{enumerate}

\end{document}